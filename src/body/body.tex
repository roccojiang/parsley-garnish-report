\documentclass[../../main.tex]{subfiles}

\begin{document}

% Burmako:
%  A metaprogramming framework (also, a metaprogramming library) is a collection of data structures and operations that allow to write metaprograms, i.e. programs that manipulate other programs as data. For example, scala.reflect which is developed and described in this dissertation is a metaprogramming framework, because it contains classes like Tree, Symbol, Type and others that represent Scala programs as data and exposes operations like prettyprinting, name resolution, typechecking and others that can manipulate this data.
% We say that metaprogramming frameworks provide the ability to reflect programs, i.e. reify program elements as data structures that can be introspected by metaprograms. For instance, scala.reflect provides facilities to reflect Scala programs. These facilities include the ability to reify definitions written in the source code as instances of class Symbol. Metaprograms can call methods like Symbol.info to introspect these definitions.

\subfile{simple-rules}%
\subfile{leftrec}%
\subfile{impl}%
\subfile{complex-rules}%

\end{document}
