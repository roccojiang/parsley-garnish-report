\documentclass[../../main.tex]{subfiles}

\begin{document}

% TODO: Update chapter titles etc. as I figure out how to properly organise these sections

\TODO{
Not sure how to order things: lint rules first, or implementation?
}

\chapter{Lint Rules in \texttt{parsley-garnish}}

\TODO{Catalogue of lint rules implemented.}

\TODO{
Categorise these -- but also somehow split into the "simple" rules and the "complex" rules.
Simple rules can consist of a single heading, containing:
% TODO: Take inspiration from DLint paper, the Scala Refactoring master's thesis, HaRe papers
* Explanation of the rule
* Simple example to show a diagnostic, and a before and after if it's fixable
* How it's implemented in the code
* Proof (if applicable)
* Limitations

Simple rule ideas:
* Overly complex parser definitions
* Manually calling implicitSymbol instead of using the implicit

Not sure how to lay out the complex rules yet -- so far this is just the left-recursion removal rule.
The other complex rule(s) will likely share implementation details with the Parser/Func representation, so work from there.
}

\section{Avoid Redefining Existing Parsers}
\TODO{
* Catch cases when user manually writes out a parser that is already defined in the library
}

\section{Simplify Complex Parsers}
\TODO{
* Apply parser laws, re-using Parser and Func representations to do cool things <- should this be a separate rule?
}

\section{Ambiguous Implicit Conversions}
\epigraph{Heroin is just one letter away from heroine, and implicit conversions are the heroine we don't deserve.}{Jamie Willis, 2024}
Implicit conversions are a powerful feature in Scala, allowing for automatic type coercion between types.
However, they are often considered a double-edged sword, as they can be easily abused, leading to unexpected behaviour and making code harder to reason about.

\TODO{
* Implicit conversions are controversial, but less evil when it comes to dsls
* from the man odersky himself: https://contributors.scala-lang.org/t/can-we-wean-scala-off-implicit-conversions/4388
* Cite Jamie's design pattern from the scala paper - pattern 0 and 2c
* See his discussion in phd thesis (nothing has been elaborated on in the masters thesis)

* have to use syntacticrule because we want to be able to provide diagnostics even if compilation fails, due to the exact problem of clashing implicits
* so actually this would've been better as lint-on-compile, to annotate the compiler error message at the exact location - we'd also crucially have more information at that stage
* but because the only other option for scalafix is semanticrules that are completely after compilation time, we can't gather even this partial information
% http://www.wartremover.org/doc/warts.html#implicitconversion
* wartremover actually has a wart for this \^, but it only targets the definition of implicit conversions, so it's not actually what we'd want out of a lint-on-compile rule
}

\section{Remove Explicit Usage of Implicit Conversions}

\section{Refactor to use Parser Bridges}
\TODO{
* This would be cool, idk if I have time though, but this should also piggyback off of Func
* the pos bridges don't actually exist, so we can ignore that case and just say its too much code synthesis
* shouldn't be too bad? idk
* indicate limitations that this will only work if the ADT is defined in the same file, in order to extend it
}

\section{Left Recursion Removal}

\chapter{Implementation}

\TODO{
Non-terminal detection. This may get reworked/renamed since it's pretty specialised for leftrec rn, and in reality it's just trying to grab all the parsers. % <- where does this go?

Other util things?
ACTUALLY NEED TO DO: import combinators if they aren't already imported
}

\section{Parser Representation}

\TODO{
% Garnishing Parsec with Parsley 2018: Parsley is a deep embedding: the language is represented by objects and behaviours are realised as methods on an abstract trait. In Parsley’s case, the deep embedding provides methods for code generation and optimisation and classes for each core combinator. The advantage of a deep embedding over a shallow embedding is that it significantly easier to optimise using pattern matching on constructors instead of bytecode.
Representation of Parsley combinators in \texttt{parsley-garnish}. Compare with approach in Scala Parsley, take cues from the 2018 paper.
* Approach to composites? Need to think about this.
  * For LeftRec: Parse ASTs into a small group of core combinators, but we also need to represent composite combinators as their own case classes -- recombine/"simplify" after analysis is concluded, it doesn't really matter if we completely change what combinators are used as long as semantic meaning is preserved.
  * For others: probably need to parse directly into composite combinators, since we don't want to destructively modify what combinators have been used.
* Optimisations: for us, the goal is human readability, so this is interesting to compare to the paper. Lots of similar stuff actually, like top-down peephole optimisations utilising parser laws (I think I do it this way? Need to double check).
  * For cleanliness to isolate boilerplate: https://blog.sumtypeofway.com/posts/introduction-to-recursion-schemes.html -- we don't have a generic traversal, but we can decouple the recursive application of a given partial function from the actual pf itself (I've called it .transform for the Parser class)
}

\section{Function Representation}
\TODO{
Abstraction built over scalafix/meta ASTs to represent functions.
Allows us to statically evaluate function composition/flipping etc, so it doesn't turn into one big mess -- again, human readability of the transformed output is the goal.
% https://hugopeters.me/posts/16/ -- should I switch to de bruijn indices? Bound variables get instantiated only at the end, don't have to worry about alpha conversion (relation to Barendregt's convention if needed??)
Abstraction is again an ADT as a lambda calculus, but with parameter lists so not everything is curried.
\^ idk, this is still a work-in-progress. Seems that there might not be enough time to uncurry the leftrec analysis so this design decision might not be super important.
Representation as a lambda calc has allocation overhead, but greatly simplifies function evaluation via beta reduction, instead of having to deal with high-level representations of compose/id (not too bad tbh) and flip (annoying).
Also attempted to make it typed but that didn't go so well with Scala's limitations on type inference.

* Extracting method arguments (alongside their types) is very painful
* Need to unify information from signature (within symbolinformation) and synthetics
  * synthetics exist in certain cases: .apply methods, showing the concrete type of a generic argument, implicit conversions
    * from https://scalacenter.github.io/scalafix/docs/developers/semantic-tree.html: SemanticTree is a sealed data structure that encodes tree nodes that are generated by the compiler from inferred type parameters, implicit arguments, implicit conversions, inferred .apply and for-comprehensions.

% TODO: FIGURE OUT ALL THE IMPORTANT CASES TO COVER:
* map, lift (implicit and explicit), zipped, (.as perhaps?)
  -- these should surely boil down into two cases: (x, y).xxx(f) and xxx(f, x, y)
  * named function literals (val)
  * named method literals (def)
  * anonymous functions i.e. lambdas
  * functions with placeholder syntax
  * apply methods of case classes - symbol will tell its a class signature so we use this as a clue to look at synthetics???
* generic bridges -- I reckon the information will probably show up in synthetics again

I think look at symbol signature first, then look at synthetics based on some heuristics (e.g. if no symbol sig -- if this happens will there even be synthetics?, if class signature)
}

\end{document}
