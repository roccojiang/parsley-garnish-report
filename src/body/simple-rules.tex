\documentclass[../../main.tex]{subfiles}

\begin{document}

% TODO: Take inspiration from DLint paper, the Scala Refactoring master's thesis, HaRe papers
% Simple rules can consist of a single heading, containing:
% * Explanation of the rule
% * Simple example to show a diagnostic, and a before and after if it's fixable
% * How it's implemented in the code
% * Proof (if applicable)
% * Limitations

\ourchapter{Idiomatic Implicits Usage}\label{sec:simple-rules}
Implicit conversions are a powerful feature in Scala, allowing users to supply an argument of one type when another is expected, to reduce boilerplate.
As noted by \textcite{willis_design_2022}, implicit conversions are particularly useful for designing \textsc{dsl}s.
In the context of parser combinators, they introduce the usage of implicit conversions to automatically lift string and character literals into parsers in the \emph{Implicit Conversions} design pattern.
This eliminates the need to explicitly wrap these elements in combinators:
\scala{string("parsley") | string("garnish")} can now be expressed as just \scala{"parsley" | "garnish"}, more closely resembling the style of a \textsc{bnf} grammar.

The \emph{Implicit Lexer} pattern is a further specialisation of this approach, hiding the boilerplate of whitespace handling entirely within a \scala{lexer} object.
This design pattern allows whitespace handling to be encapsulated as private combinators within the \scala{lexer} object, which are then made available only through implicit conversions automatically applied by the Scala compiler.

\section{Ambiguous Implicit Conversions}
Unfortunately, implicit conversions are a double-edged sword.
By their very nature, they can obscure the flow of the program, making it difficult to understand what code is doing and potentially hiding side effects or costly operations.
A downside particularly relevant to Parsley is that implicit conversions often lead to confusing error diagnostics when the compiler is unable to resolve them.

One common issue arises from ambiguous implicits when there are multiple implicit conversions in scope.
Parsley provides \scala{stringLift} and \scala{charLift} combinators in the \texttt{parsley.syntax.character} package for the \emph{Implicit Conversions} pattern,
and exposes an \scala{implicitSymbol} combinator for lexers to use in the \emph{Implicit Lexer} pattern.
The two implicit conversions cannot be used in conjunction: the \emph{Implicit Lexer} pattern is a specialisation of the former, so \scala{implicitSymbol} is meant to be a \emph{replacement} for \scala{stringLift}
For novice users, this may not be immediately apparent and it is easy to accidentally bring both sets of these implicits into scope; anecdotally, this issue has been encountered by a number of \textsc{Wacc} students at Imperial.
For example, consider the following code snippet:
\begin{minted}{scala}
val p = 'g' ~> "arnish"
p.parse("garnish")
// [error] type mismatch;
//   found   : String("arnish")
//   required: Parsley[?]
//  Note that implicit conversions are not applicable because they are ambiguous:
//   both method stringLift in object character of type (str: String): Parsley[String]
//   and method implicitSymbol in class ImplicitSymbol of type (s: String): Parsley[Unit]
//   are possible conversion functions from String("arnish") to Parsley[?]
//    val p = 'g' ~> "arnish"
//                   ^^^^^^^^
\end{minted}
%
In this instance, the compiler provides a detailed error message indicating the ambiguity between two possible implicit conversions.
However, the compiler is not always able to report such issues clearly. For instance, switching the position of the intended implicit conversion results in a less helpful message:
\begin{minted}{scala}
val p = "garnis" <~ 'h'
p.parse("garnish")
// [error] value <~ is not a member of String
//    val p = "garnis" <~ 'h'
//            ^^^^^^^^^^^
\end{minted}

\subsection{Solution}
Ideally, this issue would be addressed by implementing a lint-on-compile rule, which could annotate the compiler error message at the exact location of the issue.
If this were implemented as a compiler plugin, partial information available from the compiler stages before the error could potentially provide enough detail to identify the exact clashing implicits.
This approach would allow leveraging domain knowledge to update the error message with more useful Parsley-specific diagnostics.

Incidentally, WartRemover has a related lint rule for implicit conversions\footnote{\url{http://www.wartremover.org/doc/warts.html#implicitconversion}},
although it only targets the locations where implicit conversions are \emph{defined}, not where they are \emph{applied}.
Despite this limitation, it serves as a proof of concept demonstrating the feasibility of such an approach.

Unfortunately, Scalafix restricts usage to only syntactic rules on the bare \textsc{ast} or semantic rules that operate fully post-compilation.
Since the ambiguous implicit conversions will cause compilation failures, this lint must be implemented as a syntactic rule.
Consequently, the solution takes a different approach: examining a file's import clauses to deduce the presence of clashing implicits.

\subsection{Example}
\Cref{fig:ambiguous-implicits-example} extends the previous example to a full Scala source file following the \emph{Implicit Lexer} pattern,
but where the user has erroneously additionally imported the \scala{stringLift} implicit from the \emph{Implicit Conversions} pattern.
This results in the Scala compiler throwing an error on line 6 due to ambiguous implicits.
When run on this file, \texttt{parsley-garnish} will report a warning similar to that shown in \cref{fig:ambiguous-implicits-warning}.

\begin{figure}[htbp]
\begin{subfigure}{\textwidth}
\begin{minted}[frame=single,linenos]{scala}
object parser {
  import parsley.syntax.character.stringLift
  import lexer.implicits._
  val p = "garnis" <~ 'h'
}

import parsley.token.Lexer
object lexer {
  import parsley.token.descriptions.LexicalDesc
  private val lexer = new Lexer(LexicalDesc.plain)
  val implicits = lexer.lexeme.symbol.implicits
}
\end{minted}
\caption{A minimal Parsley program which fails to compile due to ambiguous implicits in the \texttt{parser} object.}
\label{fig:ambiguous-implicits-example}
\end{subfigure}
%
\begin{subfigure}{\textwidth}
\vspace{3ex} % TODO: ew
\begin{minted}[frame=single,fontsize=\small]{text}
warning: [AmbiguousImplicitConversions] This import may cause clashing implicit conversions:
* import parsley.syntax.character.stringLift at line 2
* import lexer.implicits._ at line 3
If this is the case, you may encounter confusing errors like 'method is not a member of String'.
To fix this, ensure that you only import a single implicit conversion.

  import lexer.implicits._
  ^^^^^^^^^^^^^^^^^^^^^^^^  
\end{minted}
\caption{The warning message reported at line 3 by the \emph{Ambiguous Implicit Conversions} lint rule.}
\label{fig:ambiguous-implicits-warning}
\end{subfigure}
\caption{Example of the \emph{Ambiguous Implicit Conversions} lint rule in action.}
\end{figure}

\subsection{Implementation}
The rule uses the following heuristics to determine if an import clause brings an implicit into scope:
\begin{itemize}
  \item An import clause of the form \scala{import parsley.syntax.character.xxx}, where \scala{xxx} is either a wildcard import or specifically contains the importee \scala{stringLift}, indicates the \emph{Implicit Conversions} pattern on strings.
  \item The import path for the \emph{Implicit Lexer} pattern is subject to how the user names their lexer object, so determining this is much more heuristic-based. The following cases are considered indicative of this pattern:
  \begin{itemize}
    \item An import containing the keyword \scala{lexer}, as well as either \scala{implicit} or \scala{implicits}.
    \item An import clause that brings in an importee called \scala{implicitSymbol}.
  \end{itemize}
\end{itemize}
%
% https://alvinalexander.com/scala/how-to-use-scala-imports-statements-anywhere-methods-blocks-scope/
Unlike Java, Scala allows import clauses to appear anywhere in source files, not just at the top of the file.
The flexibility of Scala's import system has two main repercussions:
\begin{enumerate}
  \item Imports are visible only within the scope in which they are defined. For example, the members imported by lines 2--3 in \cref{fig:ambiguous-implicits-example} are only available in the \scala{parser} object, and not visible within \scala{lexer}.
  \item Imports are resolved lexically in a top-down order within the source file, so members cannot be referenced at any point before they were imported. In \cref{fig:ambiguous-implicits-example}, the import on line 7 brings parsley's \scala{Lexer} into scope at any scope after that line, but will not be visible before that point.
\end{enumerate} 
This behaviour presents extra challenges for this lint rule, as it must be aware of the exact scope of each import to determine which implicits are in scope at any given point.

The lexically ordered management of scope can be respected by visiting all import clauses in the order they appear in the source file.
By treating the parent of an \textsc{ast} node as its enclosing scope, it is possible to use ancestor information to determine the scoping relation between two nodes.
Scoping rules can then be respected by using this information to determine which previously visited imports are still in scope at any given point in the traversal.
\begin{minted}{scala}
extension (tree: Tree) {
  def isWithinScope(scope: Tree): Boolean =
    tree.isEqual(scope) || tree.parent.exists(_.isWithinScope(scope))
}
\end{minted}
%
With this, \cref{fig:ambiguous-implicits-impl} shows the core logic of the rule, implemented as a \emph{tail-recursive} visit over the list of imports, which allows the Scala compiler to optimise the recursion into an efficient iterative loop.
The \scala{getClashingImports} method uses the heuristic rules described above to determine which implicits are brought into scope by an import.
If an import brings in both types of implicits for the first time within its enclosing scope, a patch is generated to report a diagnostic.
The traversal then continues recursively, updating its list of visited imports.

The overall entry point of the rule is thus defined as follows, where the import clauses are gathered in order via a top-down traversal of the \textsc{ast}:
\begin{minted}{scala}
override def fix(implicit doc: SyntacticDocument): Patch = {
  val orderedImports = doc.tree.collect { case i: Import => i }
  visit(orderedImports, List.empty, Patch.empty)
}
\end{minted}

\begin{figure}[htbp]
\begin{minted}{scala}
def visit(unvisited: List[Import], visited: List[Import], patches: Patch): Patch = {
  unvisited match {
    case curr :: remaining =>
      val currentScope = curr.parent.get // Parent node represents import's scope
      // Which imports are in scope at this point: lexical ordering is maintained
      // by only considering imports that have been defined in the file so far
      val alreadyInScope = visited.filter(currentScope.isWithinScope(_.parent.get))
      val currentInScope = alreadyInScope :+ curr
      val clashingImports = getClashingImports(currentInScope)

      if (getClashingImports(alreadyInScope).isEmpty && clashingImports.nonEmpty) {
        visit(remaining, currentInScope, patches +
          Patch.lint(AmbiguousImplicitConversionsLint(curr, clashingImports)))
      } else {
        // Continue traversal, discarding imports that have fallen out of scope
        visit(remaining, currentInScope, patches)
      }
    case Nil => patches
  }
}
\end{minted}
\caption{Implementation of the core logic for the \emph{Ambiguous Implicit Conversions} lint rule.}
\label{fig:ambiguous-implicits-impl}
\end{figure}

\section{Remove Explicit Usage of Implicit Conversions}

\end{document}
